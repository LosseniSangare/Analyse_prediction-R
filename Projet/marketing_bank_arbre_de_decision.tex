% Options for packages loaded elsewhere
\PassOptionsToPackage{unicode}{hyperref}
\PassOptionsToPackage{hyphens}{url}
%
\documentclass[
]{article}
\usepackage{amsmath,amssymb}
\usepackage{iftex}
\ifPDFTeX
  \usepackage[T1]{fontenc}
  \usepackage[utf8]{inputenc}
  \usepackage{textcomp} % provide euro and other symbols
\else % if luatex or xetex
  \usepackage{unicode-math} % this also loads fontspec
  \defaultfontfeatures{Scale=MatchLowercase}
  \defaultfontfeatures[\rmfamily]{Ligatures=TeX,Scale=1}
\fi
\usepackage{lmodern}
\ifPDFTeX\else
  % xetex/luatex font selection
\fi
% Use upquote if available, for straight quotes in verbatim environments
\IfFileExists{upquote.sty}{\usepackage{upquote}}{}
\IfFileExists{microtype.sty}{% use microtype if available
  \usepackage[]{microtype}
  \UseMicrotypeSet[protrusion]{basicmath} % disable protrusion for tt fonts
}{}
\makeatletter
\@ifundefined{KOMAClassName}{% if non-KOMA class
  \IfFileExists{parskip.sty}{%
    \usepackage{parskip}
  }{% else
    \setlength{\parindent}{0pt}
    \setlength{\parskip}{6pt plus 2pt minus 1pt}}
}{% if KOMA class
  \KOMAoptions{parskip=half}}
\makeatother
\usepackage{xcolor}
\usepackage[margin=1in]{geometry}
\usepackage{color}
\usepackage{fancyvrb}
\newcommand{\VerbBar}{|}
\newcommand{\VERB}{\Verb[commandchars=\\\{\}]}
\DefineVerbatimEnvironment{Highlighting}{Verbatim}{commandchars=\\\{\}}
% Add ',fontsize=\small' for more characters per line
\usepackage{framed}
\definecolor{shadecolor}{RGB}{248,248,248}
\newenvironment{Shaded}{\begin{snugshade}}{\end{snugshade}}
\newcommand{\AlertTok}[1]{\textcolor[rgb]{0.94,0.16,0.16}{#1}}
\newcommand{\AnnotationTok}[1]{\textcolor[rgb]{0.56,0.35,0.01}{\textbf{\textit{#1}}}}
\newcommand{\AttributeTok}[1]{\textcolor[rgb]{0.13,0.29,0.53}{#1}}
\newcommand{\BaseNTok}[1]{\textcolor[rgb]{0.00,0.00,0.81}{#1}}
\newcommand{\BuiltInTok}[1]{#1}
\newcommand{\CharTok}[1]{\textcolor[rgb]{0.31,0.60,0.02}{#1}}
\newcommand{\CommentTok}[1]{\textcolor[rgb]{0.56,0.35,0.01}{\textit{#1}}}
\newcommand{\CommentVarTok}[1]{\textcolor[rgb]{0.56,0.35,0.01}{\textbf{\textit{#1}}}}
\newcommand{\ConstantTok}[1]{\textcolor[rgb]{0.56,0.35,0.01}{#1}}
\newcommand{\ControlFlowTok}[1]{\textcolor[rgb]{0.13,0.29,0.53}{\textbf{#1}}}
\newcommand{\DataTypeTok}[1]{\textcolor[rgb]{0.13,0.29,0.53}{#1}}
\newcommand{\DecValTok}[1]{\textcolor[rgb]{0.00,0.00,0.81}{#1}}
\newcommand{\DocumentationTok}[1]{\textcolor[rgb]{0.56,0.35,0.01}{\textbf{\textit{#1}}}}
\newcommand{\ErrorTok}[1]{\textcolor[rgb]{0.64,0.00,0.00}{\textbf{#1}}}
\newcommand{\ExtensionTok}[1]{#1}
\newcommand{\FloatTok}[1]{\textcolor[rgb]{0.00,0.00,0.81}{#1}}
\newcommand{\FunctionTok}[1]{\textcolor[rgb]{0.13,0.29,0.53}{\textbf{#1}}}
\newcommand{\ImportTok}[1]{#1}
\newcommand{\InformationTok}[1]{\textcolor[rgb]{0.56,0.35,0.01}{\textbf{\textit{#1}}}}
\newcommand{\KeywordTok}[1]{\textcolor[rgb]{0.13,0.29,0.53}{\textbf{#1}}}
\newcommand{\NormalTok}[1]{#1}
\newcommand{\OperatorTok}[1]{\textcolor[rgb]{0.81,0.36,0.00}{\textbf{#1}}}
\newcommand{\OtherTok}[1]{\textcolor[rgb]{0.56,0.35,0.01}{#1}}
\newcommand{\PreprocessorTok}[1]{\textcolor[rgb]{0.56,0.35,0.01}{\textit{#1}}}
\newcommand{\RegionMarkerTok}[1]{#1}
\newcommand{\SpecialCharTok}[1]{\textcolor[rgb]{0.81,0.36,0.00}{\textbf{#1}}}
\newcommand{\SpecialStringTok}[1]{\textcolor[rgb]{0.31,0.60,0.02}{#1}}
\newcommand{\StringTok}[1]{\textcolor[rgb]{0.31,0.60,0.02}{#1}}
\newcommand{\VariableTok}[1]{\textcolor[rgb]{0.00,0.00,0.00}{#1}}
\newcommand{\VerbatimStringTok}[1]{\textcolor[rgb]{0.31,0.60,0.02}{#1}}
\newcommand{\WarningTok}[1]{\textcolor[rgb]{0.56,0.35,0.01}{\textbf{\textit{#1}}}}
\usepackage{graphicx}
\makeatletter
\def\maxwidth{\ifdim\Gin@nat@width>\linewidth\linewidth\else\Gin@nat@width\fi}
\def\maxheight{\ifdim\Gin@nat@height>\textheight\textheight\else\Gin@nat@height\fi}
\makeatother
% Scale images if necessary, so that they will not overflow the page
% margins by default, and it is still possible to overwrite the defaults
% using explicit options in \includegraphics[width, height, ...]{}
\setkeys{Gin}{width=\maxwidth,height=\maxheight,keepaspectratio}
% Set default figure placement to htbp
\makeatletter
\def\fps@figure{htbp}
\makeatother
\setlength{\emergencystretch}{3em} % prevent overfull lines
\providecommand{\tightlist}{%
  \setlength{\itemsep}{0pt}\setlength{\parskip}{0pt}}
\setcounter{secnumdepth}{-\maxdimen} % remove section numbering
\ifLuaTeX
  \usepackage{selnolig}  % disable illegal ligatures
\fi
\IfFileExists{bookmark.sty}{\usepackage{bookmark}}{\usepackage{hyperref}}
\IfFileExists{xurl.sty}{\usepackage{xurl}}{} % add URL line breaks if available
\urlstyle{same}
\hypersetup{
  pdftitle={Marketing\_Bank\_Arbre\_de\_decision},
  pdfauthor={ISOLA Obédi Omokayodé},
  hidelinks,
  pdfcreator={LaTeX via pandoc}}

\title{Marketing\_Bank\_Arbre\_de\_decision}
\author{ISOLA Obédi Omokayodé}
\date{2023-12-13}

\begin{document}
\maketitle

\#ARBRE DE DECISION

\begin{Shaded}
\begin{Highlighting}[]
\FunctionTok{options}\NormalTok{(}\AttributeTok{repos =} \FunctionTok{c}\NormalTok{(}\AttributeTok{CRAN =} \StringTok{"https://cran.r{-}project.org"}\NormalTok{))}\CommentTok{\# pour permettre l\textquotesingle{}execution du Knit}
\end{Highlighting}
\end{Shaded}

\begin{Shaded}
\begin{Highlighting}[]
\FunctionTok{rm}\NormalTok{(}\AttributeTok{list=}\FunctionTok{ls}\NormalTok{())}
\end{Highlighting}
\end{Shaded}

\hypertarget{chargement-des-fichiers}{%
\subsection{Chargement des fichiers}\label{chargement-des-fichiers}}

\begin{Shaded}
\begin{Highlighting}[]
\NormalTok{souscription }\OtherTok{\textless{}{-}} \FunctionTok{read.csv}\NormalTok{(}\StringTok{"D:/UTT/PROGRAMATION R/Data/Souscription.csv"}\NormalTok{, }\AttributeTok{sep=}\StringTok{";"}\NormalTok{)}
\FunctionTok{View}\NormalTok{(souscription)}
\end{Highlighting}
\end{Shaded}

\#\#Importation de la librairie de, l'arbre de decision

\begin{Shaded}
\begin{Highlighting}[]
\FunctionTok{install.packages}\NormalTok{(}\StringTok{"dplyr"}\NormalTok{)}
\end{Highlighting}
\end{Shaded}

\begin{verbatim}
## Installation du package dans 'C:/Users/LOSSENI SANGARE/AppData/Local/R/win-library/4.3'
## (car 'lib' n'est pas spécifié)
\end{verbatim}

\begin{verbatim}
## le package 'dplyr' a été décompressé et les sommes MD5 ont été vérifiées avec succés
## 
## Les packages binaires téléchargés sont dans
##  C:\Users\LOSSENI SANGARE\AppData\Local\Temp\RtmpwbOKs9\downloaded_packages
\end{verbatim}

\begin{Shaded}
\begin{Highlighting}[]
\FunctionTok{library}\NormalTok{(dplyr)}
\end{Highlighting}
\end{Shaded}

\begin{verbatim}
## Warning: le package 'dplyr' a été compilé avec la version R 4.3.2
\end{verbatim}

\begin{verbatim}
## 
## Attachement du package : 'dplyr'
\end{verbatim}

\begin{verbatim}
## Les objets suivants sont masqués depuis 'package:stats':
## 
##     filter, lag
\end{verbatim}

\begin{verbatim}
## Les objets suivants sont masqués depuis 'package:base':
## 
##     intersect, setdiff, setequal, union
\end{verbatim}

\begin{Shaded}
\begin{Highlighting}[]
\FunctionTok{install.packages}\NormalTok{(}\StringTok{"rpart"}\NormalTok{)}
\end{Highlighting}
\end{Shaded}

\begin{verbatim}
## Installation du package dans 'C:/Users/LOSSENI SANGARE/AppData/Local/R/win-library/4.3'
## (car 'lib' n'est pas spécifié)
\end{verbatim}

\begin{verbatim}
## le package 'rpart' a été décompressé et les sommes MD5 ont été vérifiées avec succés
\end{verbatim}

\begin{verbatim}
## Warning: impossible de supprimer l'installation précédente du package 'rpart'
\end{verbatim}

\begin{verbatim}
## Warning in file.copy(savedcopy, lib, recursive = TRUE): problème lors de la
## copie de C:\Users\LOSSENI
## SANGARE\AppData\Local\R\win-library\4.3\00LOCK\rpart\libs\x64\rpart.dll vers
## C:\Users\LOSSENI
## SANGARE\AppData\Local\R\win-library\4.3\rpart\libs\x64\rpart.dll: Permission
## denied
\end{verbatim}

\begin{verbatim}
## Warning: 'rpart' restauré
\end{verbatim}

\begin{verbatim}
## 
## Les packages binaires téléchargés sont dans
##  C:\Users\LOSSENI SANGARE\AppData\Local\Temp\RtmpwbOKs9\downloaded_packages
\end{verbatim}

\begin{Shaded}
\begin{Highlighting}[]
\FunctionTok{library}\NormalTok{(rpart)}
\end{Highlighting}
\end{Shaded}

\begin{verbatim}
## Warning: le package 'rpart' a été compilé avec la version R 4.3.2
\end{verbatim}

\hypertarget{les-donnuxe9es-en-ensembles-dentrauxeenement-70-et-de-test-30}{%
\subsection{les données en ensembles d'entraînement (70\%) et de test
(30\%)}\label{les-donnuxe9es-en-ensembles-dentrauxeenement-70-et-de-test-30}}

\begin{Shaded}
\begin{Highlighting}[]
  \FunctionTok{set.seed}\NormalTok{(}\DecValTok{1234}\NormalTok{)  }\CommentTok{\# Pour la reproductibilité}
\NormalTok{  indices }\OtherTok{\textless{}{-}} \FunctionTok{sample}\NormalTok{(}\DecValTok{1}\SpecialCharTok{:}\FunctionTok{nrow}\NormalTok{(souscription), }\AttributeTok{size =} \FloatTok{0.7} \SpecialCharTok{*} \FunctionTok{nrow}\NormalTok{(souscription))}
\NormalTok{  train\_data }\OtherTok{\textless{}{-}}\NormalTok{ souscription[indices, ]}
\NormalTok{  test\_data }\OtherTok{\textless{}{-}}\NormalTok{ souscription[}\SpecialCharTok{{-}}\NormalTok{indices, ]}
\end{Highlighting}
\end{Shaded}

\hypertarget{sauvegarde-et-suppression-de-la-cible-y-des-donnuxe9e-de-test}{%
\subsection{sauvegarde et suppression de la cible y des donnée de
test}\label{sauvegarde-et-suppression-de-la-cible-y-des-donnuxe9e-de-test}}

\begin{Shaded}
\begin{Highlighting}[]
\CommentTok{\#cible\_test = test\_data$y}
\CommentTok{\#test\_data = test\_data[,{-}ncol(test\_data)]}
\end{Highlighting}
\end{Shaded}

\#\#Modelisation par l'arbre de decision de la variable dependante y en
fonction des autres variable

\begin{Shaded}
\begin{Highlighting}[]
\NormalTok{arbre }\OtherTok{=} \FunctionTok{rpart}\NormalTok{(y}\SpecialCharTok{\textasciitilde{}}\NormalTok{.,}\AttributeTok{data =}\NormalTok{ train\_data, }\AttributeTok{cp=}\FloatTok{0.02}\NormalTok{)}
\NormalTok{arbre}
\end{Highlighting}
\end{Shaded}

\begin{verbatim}
## n= 2883 
## 
## node), split, n, loss, yval, (yprob)
##       * denotes terminal node
## 
##  1) root 2883 324 no (0.88761707 0.11238293)  
##    2) nr.employed>=5087.65 2508 154 no (0.93859649 0.06140351)  
##      4) duration< 567.5 2284  56 no (0.97548161 0.02451839) *
##      5) duration>=567.5 224  98 no (0.56250000 0.43750000)  
##       10) duration< 838.5 139  44 no (0.68345324 0.31654676) *
##       11) duration>=838.5 85  31 yes (0.36470588 0.63529412) *
##    3) nr.employed< 5087.65 375 170 no (0.54666667 0.45333333)  
##      6) duration< 165.5 131  24 no (0.81679389 0.18320611) *
##      7) duration>=165.5 244  98 yes (0.40163934 0.59836066)  
##       14) pdays>=15.5 176  85 yes (0.48295455 0.51704545)  
##         28) duration< 390.5 127  56 no (0.55905512 0.44094488) *
##         29) duration>=390.5 49  14 yes (0.28571429 0.71428571) *
##       15) pdays< 15.5 68  13 yes (0.19117647 0.80882353) *
\end{verbatim}

\hypertarget{affichage-de-larbre-produit-sur-notre-jeu-de-donnuxe9es-dentrainement}{%
\subsection{affichage de l'arbre produit sur notre jeu de données
d'entrainement}\label{affichage-de-larbre-produit-sur-notre-jeu-de-donnuxe9es-dentrainement}}

pour cela nous avons besoin d'importer le module plot de la librairie
rpart que nous venons de charger

\begin{Shaded}
\begin{Highlighting}[]
\FunctionTok{install.packages}\NormalTok{(}\StringTok{"rpart.plot"}\NormalTok{)}
\end{Highlighting}
\end{Shaded}

\begin{verbatim}
## Installation du package dans 'C:/Users/LOSSENI SANGARE/AppData/Local/R/win-library/4.3'
## (car 'lib' n'est pas spécifié)
\end{verbatim}

\begin{verbatim}
## le package 'rpart.plot' a été décompressé et les sommes MD5 ont été vérifiées avec succés
## 
## Les packages binaires téléchargés sont dans
##  C:\Users\LOSSENI SANGARE\AppData\Local\Temp\RtmpwbOKs9\downloaded_packages
\end{verbatim}

\begin{Shaded}
\begin{Highlighting}[]
\FunctionTok{library}\NormalTok{(}\StringTok{"rpart.plot"}\NormalTok{)}
\end{Highlighting}
\end{Shaded}

\begin{verbatim}
## Warning: le package 'rpart.plot' a été compilé avec la version R 4.3.2
\end{verbatim}

\begin{Shaded}
\begin{Highlighting}[]
\FunctionTok{rpart.plot}\NormalTok{(arbre,}\AttributeTok{main=}\StringTok{"Representation de l\textquotesingle{}arbre"}\NormalTok{)}
\end{Highlighting}
\end{Shaded}

\includegraphics{marketing_bank_arbre_de_decision_files/figure-latex/unnamed-chunk-8-1.pdf}

\#\#prediction sur les données de test

\begin{Shaded}
\begin{Highlighting}[]
\NormalTok{pred }\OtherTok{=} \FunctionTok{predict}\NormalTok{(arbre, }\AttributeTok{data=}\NormalTok{test\_data)}
\end{Highlighting}
\end{Shaded}

\#\#comparaison de la prédiction par rapport aux vrai valeur y de
l'ensemble de test

\begin{Shaded}
\begin{Highlighting}[]
\FunctionTok{length}\NormalTok{(pred)}
\end{Highlighting}
\end{Shaded}

\begin{verbatim}
## [1] 5766
\end{verbatim}

\begin{Shaded}
\begin{Highlighting}[]
\FunctionTok{length}\NormalTok{(test\_data}\SpecialCharTok{$}\NormalTok{y)}
\end{Highlighting}
\end{Shaded}

\begin{verbatim}
## [1] 1236
\end{verbatim}

\begin{Shaded}
\begin{Highlighting}[]
\FunctionTok{length}\NormalTok{(train\_data}\SpecialCharTok{$}\NormalTok{y)}
\end{Highlighting}
\end{Shaded}

\begin{verbatim}
## [1] 2883
\end{verbatim}

\begin{Shaded}
\begin{Highlighting}[]
\CommentTok{\#table(test\_data$y, pred)}

\FunctionTok{dim}\NormalTok{(train\_data)}
\end{Highlighting}
\end{Shaded}

\begin{verbatim}
## [1] 2883   21
\end{verbatim}

\begin{Shaded}
\begin{Highlighting}[]
\FunctionTok{dim}\NormalTok{(test\_data)}
\end{Highlighting}
\end{Shaded}

\begin{verbatim}
## [1] 1236   21
\end{verbatim}

\end{document}
